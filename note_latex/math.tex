\documentclass[11pt,a4paper]{article}
\usepackage[UTF8]{ctex}
\usepackage{amsmath, amssymb, amsthm}
\usepackage{enumitem}
\usepackage{geometry}
\geometry{left=2.5cm, right=2.5cm, top=2.5cm, bottom=2.5cm}

\title{带阶乘的极限的常用解决方法}


\begin{document}

\maketitle

带阶乘的极限问题在数学分析中较为常见,通常出现在涉及数列、级数或组合数学的极限计算中。解决这类问题的方法较为丰富,以下是几种常用且有效的策略:

\section*{1. 斯特林公式(Stirling's Approximation)}
适用于阶乘出现在分子或分母,且变量趋于无穷的情况。

斯特林公式:
$$
n! \sim \sqrt{2\pi n} \left( \frac{n}{e} \right)^n \quad (n \to \infty)
$$

\textbf{适用场景}:

\begin{itemize}

    \item 形如 $\lim_{n \to \infty} \frac{n!}{n^n}$
    \item 或更复杂的表达式如 $\displaystyle \lim_{n \to \infty} \frac{(2n)!}{(n!)^2 4^n}$
\end{itemize}
\textbf{优点}:可将阶乘转化为幂函数和指数函数,便于取对数或比较增长速度。

\section*{2. 比值判别法(Ratio Test 思想)}
构造相邻项的比值,考察极限行为。

设 $a_n$ 含有阶乘,考虑:
\[
\lim_{n \to \infty} \left| \frac{a_{n+1}}{a_n} \right| = L
\]
若用于极限本身(而非级数收敛性),有时可反推 $a_n \to 0$ 或 $\infty$。

\textbf{例子}:
\[
a_n = \frac{n!}{n^n} \Rightarrow \frac{a_{n+1}}{a_n} = \frac{(n+1)!}{(n+1)^{n+1}} \cdot \frac{n^n}{n!} = \left( \frac{n}{n+1} \right)^n \to \frac{1}{e} < 1 \Rightarrow a_n \to 0
\]

\section*{3. 取对数转化为求和}
利用 $\ln(n!) = \sum_{k=1}^n \ln k$,再用积分估计(如欧拉–麦克劳林公式或积分比较)。

\textbf{积分估计}:
\[
\int_1^n \ln x \, dx < \ln(n!) < \int_1^{n+1} \ln x \, dx
\]
可推得 $\ln(n!) \sim n \ln n - n$,这其实也是斯特林公式的对数形式。

\textbf{适用场景}:阶乘出现在指数或对数中,如 $\displaystyle \lim_{n \to \infty} \sqrt[n]{n!}$

\section*{4. 夹逼定理(Squeeze Theorem)}
通过不等式对阶乘进行放缩。

\textbf{例如}:
\[
\left( \frac{n}{2} \right)^{n/2} < n! < n^n
\]
或利用组合数不等式:
\[
\binom{2n}{n} \le 4^n \Rightarrow \frac{(2n)!}{(n!)^2} \le 4^n
\]

\section*{5. 利用已知极限或级数展开}
有些极限可通过幂级数(如 $e^x = \sum \frac{x^n}{n!}$) 的性质间接求出。

\textbf{例子}:
\[
\lim_{n \to \infty} \frac{n^n}{n!} = \infty \quad \text{(因为 } e^n = \sum_{k=0}^\infty \frac{n^k}{k!} > \frac{n^n}{n!} \text{)}
\]

\section*{6. 伽马函数延拓(较少用)}
对非整数阶乘,可使用 $\Gamma(n+1) = n!$,再用其积分表达式或渐近展开,但通常在高等分析中使用。

\section*{总结:选择策略的依据}

\begin{center}
\begin{tabular}{|l|l|}
\hline
情况 & 推荐方法 \\
\hline
$n \to \infty$,阶乘在分子/分母 & 斯特林公式 \\
含 $\frac{a_{n+1}}{a_n}$ 结构 & 比值法 \\
阶乘在根号或指数中 & 取对数 + 积分估计 \\
可构造上下界 & 夹逼定理 \\
与 $e^x$、$\sin x$ 等级数相关 & 利用泰勒展开 \\
\hline
\end{tabular}
\end{center}

\medskip

如你有具体的极限表达式,我可以帮你选择最合适的方法并详细计算。

\end{document}