\documentclass[10pt]{ctexart}
\usepackage{amsmath}
\usepackage{tikz}
\usepackage{multirow} % 确保导言区已加载
\usepackage{siunitx}
\usepackage{array}     % 支持调整行高
\usepackage{caption}   % 更好的标题控制(可选)
\usepackage[UTF8]{ctex}
\setCJKmainfont[
  BoldFont = Noto Serif CJK SC Bold,
  ItalicFont = Noto Serif CJK SC
]{Noto Serif CJK SC}

\usepackage[a3paper,margin=3.5cm]{geometry}
\usepackage{setspace}
\usepackage{titlesec}
\usepackage{graphicx} % ← 别忘了加载 graphicx!
\usepackage{hyperref}
\usepackage{enumerate}

\onehalfspacing
\setlength{\parindent}{2em}
\setlength{\parskip}{0pt}

\titleformat{\section}{\bfseries\Large\centering}{\thesection}{1em}{}
\titleformat{\subsection}{\bfseries\normalsize}{\thesubsection}{0.8em}{}

\hypersetup{colorlinks=true, linkcolor=black, citecolor=black, urlcolor=black}
% ===== 题目与作者信息(自行修改) =====
\title{{\texttt{历史笔记}}\\\textbf{中国抗日战争}}
\author{%
  % 湖南大学\quad 机械与运载工程学院\\
  705\quad 周全景 
}
\date{\today}

\begin{document}
\maketitle
\section{背景}
近代日本对中国的侵略,1894年甲午战争即是开端
$$
\text{中国抗日战争}
\left\{
\begin{array}{l}
\text{九一八事变,日本}\begin{cases}
  \text{资本主义}\rightarrow\text{金融危机}\rightarrow\text{对外扩张}\rightarrow\text{金融危机}\xrightarrow{\text{资源有限}}\text{以战养战} \\
  \text{落后性}
\end{cases} \\
\text{华北事变} \\
\text{七七事变,急剧向关内增兵} \\
\text{八一三事变}
\end{array}
\right\}
\text{日本总兵力不超过一百万,然而武器装备领先很多}
$$
\\
\quad\quad19361212,西安事变,国民党开始联共抗日

\section{国民党领导抗日战争为何不能取得胜利}
\subsection{正面战场}
\begin{itemize}
  \item 战略防御阶段以国军为主要作战对象
  \item 国民党军正面战场担负主要作战任务,例子:淞沪会战
        \begin{figure}[htbp]
          \centering
          \includegraphics[width=0.8\textwidth]{沪淞会战.jpg}
          \caption{沪淞会战}
          \label{fig:example}
        \end{figure}
\end{itemize}


\subsection{战略撤退:以空间换时间}
  \begin{itemize}
    \item 工业,企业,学校迁移到内地
    \item 焦土政策$
        \left\{
        \begin{array}{l}
        \text{花园口决堤} \\
        \text{193811文夕大火}
        \end{array}
        \right\}
        \textbf{依托军队而非普通老百姓}
        $
  \end{itemize}

\subsection{国际援助}
\begin{itemize}
  \item 英法忙于欧洲战场,慕尼黑协定
  \item 美国中立,售卖军火、物资
  \item 苏联支持中国,为维护国土安全
\end{itemize}

\subsection{战略相持阶段:分共、反共}
\begin{itemize}
  \item 日本对国民党:政治诱降为主,军事打击为辅
  \item 国民党:重申持久抗战,但是内外政策发生重大变化,1939.1,“回归卢沟桥事变前状态”,“防共委员会”,1940皖南事变
\end{itemize}

\subsection{国民党内部的腐化(自上而下)}
嫡系部队大为发展,非中央军饱受打压和歧视。

\textbf{非中央军叛逃,成为伪军,对抗共产党游击队}

\textbf{只考虑自己的军队(地位,权力,财富之源)而无心抗日(以汤恩伯为代表)}

\subsection{国民党军的蜕化}

\begin{table}[h!]
\centering
\caption{具体体现}
\label{tab:example}
\renewcommand{\arraystretch}{1.4}  % ← 增大行高(1.4倍)
\small % 或 \normalsize(默认),甚至 \large(慎用)
\begin{tabular}{|>{\raggedright\arraybackslash}p{3.5cm}|>{\raggedright\arraybackslash}p{3cm}|>{\raggedright\arraybackslash}p{5cm}|}
\hline
\multicolumn{2}{|c|}{\textbf{对象}} & \textbf{缺陷} \\
\hline
\multirow{3}{=}{军官} & 非中央军 & 非中央军高级将领是军阀,缺少军事素质,作战时往往保存实力,缺乏协同意识。 \\
\cline{2-3}
& 中央军 & 中央军高级将领多由黄埔军校短期培训出身,实战经验不足,指挥能力有限。 \\
\hline
\multicolumn{2}{|l|}{士兵} & 来源于强制征兵,待遇极差,食物短缺,许多士兵营养不良,士气低落,逃亡现象严重。 \\
\hline
\end{tabular}
\end{table}
\subsection{国民党经济无能}
特别是自从广州被日本占领之后,物价飞涨,财政赤字上涨。然而\textbf{官僚资本急剧膨胀}
\subsection{国统区军民关系恶化}
改收实物税(征收田赋,征购粮食),\textbf{按照国民党需求征收,引起农民反感}

\end{document}
