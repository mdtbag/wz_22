\documentclass[12pt]{ctexart}

% ===== 基本排版设置 =====
\usepackage[a4paper,margin=2.5cm]{geometry}
\usepackage{setspace}
\usepackage{graphicx}
\usepackage{titlesec}
\usepackage{hyperref}
\usepackage{enumerate}
\usepackage{chronosys}

\onehalfspacing                 % 1.5倍行距
\setlength{\parindent}{2em}     % 首行缩进
\setlength{\parskip}{0.5em}

\titleformat{\section}{\bfseries\Large}{\thesection}{1em}{}
\titleformat{\subsection}{\bfseries\normalsize}{\thesubsection}{0.8em}{}
% ===== 题目与作者信息(自行修改) =====
\title{历史笔记:中国革命新道路的开辟}
\author{%
  xxx
}
\date{\today}

\begin{document}
\maketitle

% \begin{abstract}
% 21世纪的中国正处在能源结构深度调整与高质量发展并行的关键阶段,“碳达峰、碳中和”目标把能源与动力领域推到了时代舞台的中央。作为刚刚踏入大学校园的能源与动力专业学生,我一方面亲眼看到身边生活因能源技术而便利,另一方面也能够从新闻、社会讨论中感受到雾霾、极端天气、资源约束等问题带来的压力。本文试图在梳理当代能源发展背景的基础上,从历史维度和社会维度讨论能源与动力专业大学生应当承担的历史责任与社会责任,并结合我作为大一新生的真实处境,思考在知识储备尚不充分的阶段,可以从哪些具体行动做起,避免把“责任”停留在口号层面。文章旨在说明:责任不是抽象的大词,而是嵌在我们每天的学习选择与生活细节之中。
% \end{abstract}

% \noindent\textbf{关键词:} 能源与动力专业;历史责任;社会责任;碳中和;大学生

\section{国民党的新军阀统治}

\begin{figure}[htbp]
  \centering
  \includegraphics[width=1\textwidth]{清党对国民党的影响.png}
  \caption{国民党“清党”造成的影响}
  \label{fig:example}
\end{figure}

\subsection{动摇自身根基}
基层党组织由中国共产党建立

以“打斗”和武力武装打压,容易产生偏差,导致党员流失

$\textbf{思考:乡村中本被打压的土豪劣绅会如何活动?} \textbf{可能性:投机加入国民党,污蔑之风盛行}$

$\textbf{思考:军阀会有何反应?} \textbf{可能性:为避免蒋介石插手地方,地方强势派火速发展“党员”}$

\newpage

\subsection{用枪指挥党,应对组织瘫痪}
$\textbf{依靠军事力量维护自身统治,蜕变为新军阀}$
\begin{enumerate}

\item 与德国合作:

  \begin{itemize}
    \item 派来顾问,等级越来越高
    \item 中国一度成为德国最大军火出口地
  \end{itemize}
\item 与江浙财阀往来密切
\end{enumerate}

\subsection{国民党组织的军事化}

\begin{figure}[htbp]
  \centering
  \includegraphics[width=1.1\textwidth]{蒋介石本人:军事独裁者.png}
  \caption{蒋介石本人:军事独裁者}
\end{figure}

1934.02,新生活运动,目的是实现全民军事化,$\textbf{类似希特勒}$,为了整合社会,处理矛盾,对付日本

\begin{figure}[htbp]
  \centering
  \includegraphics[width=1\textwidth]{蒋介石政府难以实现废除封建土地所有制:平均地权.png}
  \caption{蒋介石政府难以实现废除封建土地所有制:平均地权}
\end{figure}
\newpage
\subsection{军阀混战,蒋介石党政军制衡的策略}
\begin{figure}[htbp]
  \centering
  \includegraphics[width=0.8\textwidth]{蒋介石.png}
  \caption{蒋介石政府难以实现废除封建土地所有制:平均地权}
\end{figure}

裁军会议:“裁战斗力最差的军队”,也就是指军阀的军队

1929-1930,内战,中原大战

\newpage
\section{中国共产党的反应}

\subsection{中国共产党汲取的教训}
最初工人运动强大,对抗军阀有信心

信心来源:二月革命到十月革命,苏联经验

\begin{itemize}
  \item 1927.07.12,中共中央改组,临时中央委员会
      \subitem 军事暴动
      \subitem 秋收暴动
      \subitem 翟秋白“革命潮流正在高涨”,“不断革命论”,“左倾盲动主义”导致党员损失惨重
  \item 1928.06.18 ~ 07.11,中共六大指出,处于资产阶级革命阶段而非社会主义革命阶段,要争取群众而不是现在暴动
\end{itemize}

\bigskip
\noindent\rule{\textwidth}{0.4pt}
\section*{\textcolor{blue}{本次新添内容(2025年12月09日)}}
\subsection{农村革命根据地的开辟}
\vspace*{1cm}
  \startchronology[startyear=1927,stopyear=1936]
  \chronoevent{1927}{三湾改编}
  \chronoevent{1928}{朱毛会师}
  \chronoevent{1929}{古田会议}
  \chronoevent{1931}{{\textcolor{red}{苏维埃第一次全国代表大会}}}
  \chronoevent{1930}{李立三左倾路线}
  \chronoevent{1933}{中央政治局迁至井冈山}
  \chronoevent{1934}{长征开始}
  \chronoevent{1935}{遵义会议}
  \chronoevent{1936}{打通国际联系付出巨大损失}
  \centering{\textbf{历史事件一览图}}
  \stopchronology
  \vspace*{1cm}
  \begin{description}
    \item[1927年9月] 三湾改编\begin{enumerate}
      \item 千人师缩编为团
      \item 连上建士兵委员会,官兵一律平等
      \item 各级部队建党组织\begin{itemize}
        \item 班排设小组
        \item 连建支部
        \item 团设党委
      \end{itemize}
    一切重大事务经党组织集体讨论决定,改变了军队风气
    \end{enumerate}
    \item[1928年03月] 三月失败\\周鲁错传指示,年关暴动,井冈山兵力空虚被占领,可见\textcolor{red}{左倾冒险主义和盲动主义对革命事业的危害性}
    \item[1928年04月28日] 朱毛会师
    \item[1928年05月] 湘赣边界一大\\农村条件艰苦,“红旗到底能打多久”
    \item[1928年06月] 龙源口大捷
    \item[1928年08月] 八月失败\\军事冒进与极端民主化问题,士兵委员会可以否决朱、陈意见
    \item[1929年12月] 古田会议\\\textbf{思想建军,政治建军}(前委是否应掌有军权,军队为谁服务)\\\noindent\textbf{红军的使命任务:}\begin{enumerate}
      \item 发动群众斗争,实行土地革命,建立政权
      \item 游击战争,武装人民
    \end{enumerate} 
    土地改革:以劳动力数量分配土地$\rightarrow$以人口分配土地  
    \item[1931年11月] 苏维埃第一次全国代表大会瑞金召开
  \end{description}
\subsection{围剿与反围剿}
  \begin{description}
    \item[19300611] 蒋冯阎战爆发时机,李立三意图攻占湖北武汉大城市
    \item[193009] 三届六中全会纠正李立三左倾冒险主义错误
    \item[19300619] 组建红一方面军
    \item[193010] 南昌集结大量国民党军$\leftrightarrow$罗坊会议,诱敌深入
    \item[19330107] 中央政治局撤离上海,毛泽东失去军事领导权\\土地政策犯“左”的错误,把赤白对立绝对化,过“左”政策
    
    \centering{\noindent\textbf{长征前兆}}
    \begin{itemize}
      \item 国民党军修筑碉堡,被迫由运动战转为阵地战  
      \item 被包围,物资极度缺乏
      \item 人口下降,扩红质量下降
    \end{itemize}    
  \end{description}
\section{长征:中国革命战略性转移}
  \begin{description}
    \item[193410] 湘江战役\\\textcolor{red}{人数损失过半}\\\textbf{思考:红军人数损失过半,长征成功是蒋介石放水,意图削弱军阀吗?}\\\textbf{答:}军阀与蒋矛盾深,相互怀疑;蒋财力人力有限;红军依靠地势取胜......
  \end{description}
\centering{\large{遵义会议}}
  \begin{description}
    \item[会议时间] 1935 年 1 月 15 日至 17 日(长征途中,贵州省遵义市)。
    \item[主要内容]
      \begin{itemize}
        \item 批判了博古、李德在第五次反“围剿”和长征初期的“左”倾军事错误;
        \item 肯定了毛泽东关于红军作战的基本原则;
        \item 增选毛泽东为中央政治局常委,取消博古、李德的最高军事指挥权;
        \item 决定由张闻天(洛甫)接替博古负总责,毛泽东、周恩来、王稼祥组成“三人军事指挥小组”。
      \end{itemize}

    \item[历史意义]
      \begin{itemize}
        \item \textbf{转折点}:是中国共产党历史上一个生死攸关的转折点;
        \item \textbf{独立自主}:标志着党从幼年走向成熟,开始独立自主地解决中国革命问题;
        \item \textbf{领导核心}:事实上确立了毛泽东在党中央和红军的领导地位;
        \item \textbf{路线纠正}:结束了“左”倾教条主义在中央的统治,开启了正确路线的新阶段。
      \end{itemize}
  \end{description}

  \vspace{1em}
  \noindent\footnotesize\textit{注:遵义会议是中共在未与共产国际联系下首次独立作出重大决策,为长征胜利和中国革命新局面奠定基础。}
\end{document}
